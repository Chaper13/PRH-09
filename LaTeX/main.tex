% ------------------------------------------------------------------------
% ------------------------------------------------------------------------
% Modelo de TCC baseado nas normas ABNT (NBR 14724, 10520, 6023, etc.)
% ------------------------------------------------------------------------
% ------------------------------------------------------------------------

\documentclass[
    % -- opções da classe memoir --
    12pt,               % Tamanho da fonte (NBR 14724)
    openright,          % Capítulos começam em página ímpar (frente)
    twoside,            % Impressão frente e verso
    a4paper,            % Tamanho do papel
    % -- opções do pacote babel --
    english,            % Idioma adicional para hifenização
    brazil              % O último idioma é o principal do documento
]{abntex2}

% ---
% Pacotes fundamentais 
% ---
\usepackage{lmodern}            % Usa a fonte Latin Modern
\usepackage[T1]{fontenc}        % Seleção de códigos de fonte.
\usepackage[utf8]{inputenc}     % Codificação do arquivo (inclui acentos)
\usepackage{indentfirst}        % Indenta o primeiro parágrafo de cada seção.
\usepackage{color}              % Controle das cores
\usepackage{graphicx}           % Inclusão de gráficos
\usepackage{microtype}          % Para melhorias de justificação
\usepackage{booktabs}           % Para tabelas estilo IBGE (linhas horizontais apenas)

% ---
% Pacotes de citações (NBR 10520 e NBR 6023)
% ---
\usepackage[alf]{abntex2cite}   % Citações sistema ALFABÉTICO (Autor, Data)

% ---
% Informações de dados para CAPA e FOLHA DE ROSTO (NBR 14724)
% ---
\titulo{Título do Seu Trabalho: Subtítulo se houver}
\autor{Seu Nome Completo}
\local{Rio de Janeiro}
\data{2024}
\orientador{Prof. Dr. Nome do Orientador}
\instituicao{%
  Universidade Federal do Rio de Janeiro -- UFRJ
  \par
  Faculdade de XXXXX
  \par
  Programa de Graduação em XXXXX}
\tipotrabalho{Trabalho de Conclusão de Curso (Graduação)}
% O preambulo deve conter o tipo do trabalho, o objetivo, 
% o nome da instituição e a área de concentração 
\preambulo{Trabalho de Conclusão de Curso apresentado ao Curso de [Nome do Curso] da [Instituição], como requisito parcial para obtenção do grau de Bacharel.}

% ---
% Configurações de aparência do PDF final
% ---
\definecolor{blue}{RGB}{41,5,195}
\makeatletter
\hypersetup{
    pdftitle={\@title}, 
    pdfauthor={\@author},
    pdfsubject={\imprimirpreambulo},
    pdfcreator={LaTeX with abnTeX2},
    colorlinks=true,       % false: box links; true: colored links
    linkcolor=blue,        % cor dos links internos
    citecolor=blue,        % cor dos links para a bibliografia
    filecolor=magenta,      % cor dos links para arquivos
    urlcolor=blue,
    bookmarksdepth=4
}
\makeatother

% ---
% Compila o índice
% ---
\makeindex

% ------------------------------------------------------------------------
% INÍCIO DO DOCUMENTO
% ------------------------------------------------------------------------
\begin{document}

% Retira espaço extra obsoleto entre as frases.
\frenchspacing 

% ----------------------------------------------------------
% ELEMENTOS PRÉ-TEXTUAIS (NBR 14724)
% ----------------------------------------------------------
\pretextual

% Capa
\imprimircapa

% Folha de rosto
\imprimirfolhaderosto

% ---
% Folha de aprovação (Exemplo)
% ---
\begin{folhadeaprovacao}
  \begin{center}
    {\ABNTEXchapterfont\large\imprimirautor}

    \vspace*{\fill}\vspace*{\fill}
    \begin{center}
      \ABNTEXchapterfont\bfseries\Large\imprimirtitulo
    \end{center}
    \vspace*{\fill}
    
    \hspace{.45\textwidth}
    \begin{minipage}{.5\textwidth}
        \imprimirpreambulo
    \end{minipage}%
    \vspace*{\fill}
   \end{center}
        
   Trabalho aprovado. \imprimirlocal, 22 de dezembro de 2024:

   \assinatura{\textbf{\imprimirorientador} \\ Orientador} 
   \assinatura{\textbf{Prof. Convidado 1} \\ UFRJ}
   \assinatura{\textbf{Prof. Convidado 2} \\ UFRJ}
   %\assinatura{\textbf{Professor} \\ Convidado 3}
      
   \begin{center}
      \vspace*{0.5cm}
      {\large\imprimirlocal}
      \par
      {\large\imprimirdata}
      \vspace*{1cm}
   \end{center}
\end{folhadeaprovacao}

% ---
% Dedicatória
% ---
\begin{dedicatoria}
   \vspace*{\fill}
   \centering
   \noindent
   \textit{Este trabalho é dedicado às crianças adultas que,\\
   quando pequenas, sonharam em se tornar cientistas.} \vspace*{1.5cm}
\end{dedicatoria}

% ---
% Agradecimentos
% ---
\begin{agradecimentos}
Agradeço primeiramente ao meu orientador...

Aos meus pais...
\end{agradecimentos}

% ---
% Epígrafe
% ---
\begin{epigrafe}
    \vspace*{\fill}
	\begin{flushright}
		\textit{``Não vos amoldeis às estruturas deste mundo, \\
		mas transformai-vos pela renovação da mente.''\\
		(Romanos 12:2)}
	\end{flushright}
\end{epigrafe}

% ---
% RESUMO (NBR 6028)
% ---
\setlength{\absparsep}{18pt} % ajusta o espaçamento dos parágrafos do resumo
\begin{resumo}
 Este é o resumo do trabalho. Segundo a NBR 6028, deve ser um parágrafo único, conciso, contendo objetivo, método, resultados e conclusões. Deve ter entre 150 e 500 palavras para teses e dissertações/TCCs.
 
 \textbf{Palavras-chave}: Latex. ABNT. Editoração de texto.
\end{resumo}

% ---
% ABSTRACT (Em inglês)
% ---
\begin{resumo}[Abstract]
 \begin{otherlanguage*}{english}
   This is the abstract in English.
 
   \textbf{Keywords}: Latex. ABNT. Text editing.
 \end{otherlanguage*}
\end{resumo}

% ---
% Listas de ilustrações e tabelas
% ---
\pdfbookmark[0]{\listfigurename}{lof}
\listoffigures*
\cleardoublepage

\pdfbookmark[0]{\listtablename}{lot}
\listoftables*
\cleardoublepage

% ---
% Sumário (NBR 6027)
% ---
\pdfbookmark[0]{\contentsname}{toc}
\tableofcontents*
\cleardoublepage

% ----------------------------------------------------------
% ELEMENTOS TEXTUAIS
% ----------------------------------------------------------
\textual

% ----------------------------------------------------------
% Introdução
% ----------------------------------------------------------
\chapter{Introdução}
Esta é a introdução do trabalho. Aqui você apresenta o tema, delimitação, problema e objetivos.

Exemplo de citação indireta conforme NBR 10520: O uso de \LaTeX\ facilita a vida acadêmica \cite{nbr14724}.

Exemplo de citação direta curta (até 3 linhas): ``As citações diretas, no texto, de até três linhas, devem estar contidas entre aspas duplas'' \cite[p. 2]{nbr10520}.

% ----------------------------------------------------------
% Desenvolvimento
% ----------------------------------------------------------
\chapter{Desenvolvimento}
Aqui começa a parte principal do trabalho (Revisão Bibliográfica, Metodologia, etc).

\section{Seção Secundária (NBR 6024)}
A numeração das seções é feita automaticamente.

\subsection{Exemplo de Tabela (Norma IBGE)}
Conforme as normas do IBGE (arquivo \texttt{liv23907.pdf}), tabelas não devem ter traços verticais que as fechem lateralmente. Use o pacote \texttt{booktabs}:

\begin{table}[htb]
\centering
\caption{Exemplo de Tabela estilo IBGE}
\label{tab:ibge_exemplo}
\begin{tabular}{llr} % l=left, r=right
    \toprule
    \textbf{Item} & \textbf{Descrição} & \textbf{Quantidade} \\
    \midrule
    Papel & A4 Branco & 500 \\
    Caneta & Azul Esferográfica & 50 \\
    Lápis & Preto HB & 100 \\
    \bottomrule
\end{tabular}
\fonte{Elaborado pelo autor (2024).}
\end{table}

\section{Citações e Referências}
Para citar, use o comando \texttt{\textbackslash cite\{chave\}}. Para citar autor no texto, use \texttt{\textbackslash citeonline\{chave\}}. Exemplo: Segundo \citeonline{nbr6023}, as referências devem ser alinhadas à esquerda.

% ----------------------------------------------------------
% Conclusão
% ----------------------------------------------------------
\chapter{Conclusão}
Parte final do texto, onde se apresentam as conclusões correspondentes aos objetivos e hipóteses.

% ----------------------------------------------------------
% ELEMENTOS PÓS-TEXTUAIS
% ----------------------------------------------------------
\postextual

% ----------------------------------------------------------
% Referências Bibliográficas (NBR 6023)
% ----------------------------------------------------------
\bibliography{ref} % Chama o arquivo ref.bib

% ----------------------------------------------------------
% Apêndices (Opcional)
% ----------------------------------------------------------
\begin{apendicesenv}
\partapendices
\chapter{Primeiro Apêndice}
Texto do apêndice (elaborado pelo autor).
\end{apendicesenv}

% ----------------------------------------------------------
% Anexos (Opcional)
% ----------------------------------------------------------
\begin{anexosenv}
\partanexos
\chapter{Primeiro Anexo}
Texto do anexo (documento não elaborado pelo autor).
\end{anexosenv}

\end{document}