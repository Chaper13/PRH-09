% !TeX root = main.tex
\chapter{Fundamentos da Propagação de Ondas}

\setcounter{section}{1}
\section{Equação da Onda Tridimensional em Coordenadas Cartesianas}

O objetivo desta seção é estabelecer as equações fundamentais que governam a propagação de ondas em meios elásticos tridimensionais, 
preparando o terreno para a formulação das equações de onda associadas aos diferentes modos físicos de propagação.

A seguir, as conhecidas equações fundamentais da elastodinâmica são resumidas.
Isso permite que a nomenclatura seja definida.
Apenas os aspectos que são usados diretamente nos Capítulos 6 e 7 são discutidos.
Nesta seção, assume-se primeiramente um meio elástico homogêneo isotrópico.
O amortecimento histerético é introduzido em uma etapa posterior.

\subsection{Equação de Movimento em Deformação Volumétrica e em Deformações de Rotação}

Para uma excitação harmônica com frequência angular $\omega$, assumimos que o campo de deslocamentos varia senoidalmente no tempo. 
O componente de deslocamento na direção $x$, por exemplo, pode ser expresso na forma complexa como:

\begin{equation}
    u(t) = U e^{i \omega t} \label{eq:deslocamento_complexo}
\end{equation}

Onde:

\begin{itemize}
    \item $u(t)$ é o deslocamento instantâneo no tempo $t$;
    \item $U$ é a amplitude do deslocamento (independente do tempo);
    \item $\omega$ é a frequência angular da excitação.
\end{itemize}

A representação complexa é utilizada por conveniência matemática, sendo o deslocamento físico obtido a partir da parte real da expressão.

Para encontrar o termo de inércia utilizado nas equações de equilíbrio dinâmico, derivamos o deslocamento em relação ao tempo duas vezes:

1. Velocidade:
$$\dot{u} = \frac{d}{dt}(U e^{i \omega t}) = i \omega U e^{i \omega t} = i \omega u$$

2. Aceleração:
$$\ddot{u} = \frac{d}{dt}(i \omega U e^{i \omega t}) = (i \omega)^2 U e^{i \omega t} = -\omega^2 U e^{i \omega t}$$

Como $u = U e^{i \omega t}$, temos a relação fundamental para movimento harmônico:
\begin{equation}
    \ddot{u} = -\omega^2 u \label{eq:relacao_aceleracao}
\end{equation}

\subsubsection*{Aplicação nas Equações de Equilíbrio}

Com a hipótese de excitação harmônica, a dependência temporal é eliminada,
e o problema passa a ser formulado em termos das amplitudes espaciais dos campos de deslocamento e tensão.

Ao substituir o termo de inércia da segunda lei de Newton ($F = m \cdot a \rightarrow \rho \ddot{u}$) pela relação encontrada na \eqref{eq:relacao_aceleracao},
as equações de equilíbrio dinâmico tridimensional para um meio com densidade de massa $\rho$ tornam-se:

\begin{subequations}
    \label{eq:equilibrio_dinamico}
    \begin{align}
        \sigma_{x,x} + \tau_{xy,y} + \tau_{xz,z} &= -\rho \omega^2 u \label{eq:eq_x} \\
        \tau_{yx,x} + \sigma_{y,y} + \tau_{yz,z} &= -\rho \omega^2 v \label{eq:eq_y} \\
        \tau_{zx,x} + \tau_{zy,y} + \sigma_{z,z} &= -\rho \omega^2 w \label{eq:eq_z}
    \end{align} 
\end{subequations}

A notação de subscrito após uma vírgula é utilizada para denotar uma derivada parcial. 
Dessa forma, o termo $\sigma_{x,x}$ representa a derivada da tensão normal em relação à coordenada $x$ ($\partial \sigma_x / \partial x$), 
enquanto $\tau_{xy,y}$ representa a derivada da tensão de cisalhamento em relação a $y$ ($\partial \tau_{xy} / \partial y$). 
As amplitudes de todas as tensões e deslocamentos são funções das coordenadas espaciais $x, y$ e $z$.

No regime de pequenas deformações, as relações cinemáticas entre deslocamentos e deformações específicas são dadas por:

\begin{subequations}
    \label{eq:deformacao_deslocamento}
    \begin{align}
        \epsilon_x &= u_{,x} \\
        \epsilon_y &= v_{,y} \\
        \epsilon_z &= w_{,z} \\
        \gamma_{xy} &= u_{,y} + v_{,x} \\
        \gamma_{xz} &= u_{,z} + w_{,x} \\
        \gamma_{yz} &= v_{,z} + w_{,y}
    \end{align}
\end{subequations}

As amplitudes das componentes das deformações normais e de cisalhamento são denotadas por $\epsilon$ e $\gamma$, respectivamente. 
A seguir, apresenta-se a Lei de Hooke na forma invertida, com as deformações expressas em função das tensões.

\begin{subequations}
    \label{eq:lei_hooke}
    \begin{align}
        \epsilon_x &= \frac{1}{E} \left(\sigma_x - \nu \sigma_y - \nu \sigma_z\right) \\
        \epsilon_y &= \frac{1}{E} \left(\sigma_y - \nu \sigma_x - \nu \sigma_z\right) \\
        \epsilon_z &= \frac{1}{E} \left(\sigma_z - \nu \sigma_x - \nu \sigma_y\right) \\
        \gamma_{xy} &= \frac{\tau_{xy}}{G} \\
        \gamma_{xz} &= \frac{\tau_{xz}}{G} \\
        \gamma_{yz} &= \frac{\tau_{yz}}{G}
    \end{align}
\end{subequations}

Onde o módulo de cisalhamento $G$ é relacionado ao módulo de elasticidade $E$ e ao coeficiente de Poisson $\nu$ por:

\begin{equation}
    \label{eq:G}
    G = \frac{E}{2(1 + \nu)}
\end{equation}

Para a consideração de condições de contorno em elementos infinitesimais, 
define-se o vetor de tração superficial com amplitudes de componente $t_x, t_y, t_z$.
Essas expressões serão fundamentais na imposição de condições de contorno em superfícies livres ou interfaces solo–estrutura.


\begin{subequations}
    \label{eq:tracao_superficial}
    \begin{align}
        t_x &= n_x \sigma_x + n_y \tau_{xy} + n_z \tau_{xz} \\
        t_y &= n_x \tau_{yx} + n_y \sigma_y + n_z \tau_{yz} \\
        t_z &= n_x \tau_{zx} + n_y \tau_{zy} + n_z \sigma_z
    \end{align}
\end{subequations}

Embora completas, as equações apresentadas formam um sistema fortemente acoplado de 15 componentes (tensões, deformações e deslocamentos),
cuja solução direta é pouco transparente do ponto de vista físico.
Para revelar os mecanismos fundamentais de propagação de ondas, introduz-se uma mudança de variáveis cinemáticas.

\begin{equation}
    \label{eq:def_e}
    e = u_{,x} + v_{,y} + w_{,z}
\end{equation}

\begin{subequations}
    \label{eq:def_omega}
    \begin{align}
        \Omega_x &= \frac{1}{2}(w_{,y} - v_{,z}) \\
        \Omega_y &= \frac{1}{2}(u_{,z} - w_{,x}) \\
        \Omega_z &= \frac{1}{2}(v_{,x} - u_{,y})
    \end{align}
\end{subequations}

Note que essas variáveis não introduzem novas hipóteses físicas, tratando-se apenas de uma mudança de descrição do campo de deslocamentos.

Fisicamente, a variável $e$ descreve mudanças de volume sem mudança de forma (associada às ondas primárias P),
enquanto as componentes de $\Omega$ descrevem rotações de corpo rígido de um elemento infinitesimal sem mudança de volume (associadas às ondas secundárias S). 

Como a rotação representa movimento de corpo rígido local, sem criação ou destruição de volume, o campo rotacional satisfaz naturalmente uma condição de divergência nula.

\begin{equation}
    \label{eq:continuidade_omega}
    \Omega_{x,x} + \Omega_{y,y} + \Omega_{z,z} = 0
\end{equation}

Essa decomposição permite a obtenção de equações de onda desacopladas, que serão exploradas nas seções seguintes.

Utilizando as novas variáveis cinemáticas, as três equações de movimento originais podem ser reescritas em função da deformação volumétrica $e$ e das rotações $\Omega$.
Substituindo as relações tensão-deformação e deformação-deslocamento nas equações de equilíbrio, obtém-se:

\begin{subequations}
    \label{eq:movimento_desacoplado}
    \begin{align}
        (\lambda + 2G)e_{,x} + 2G(\Omega_{y,z} - \Omega_{z,y}) &= -\rho \omega^2 u \\
        (\lambda + 2G)e_{,y} + 2G(\Omega_{z,x} - \Omega_{x,z}) &= -\rho \omega^2 v \\
        (\lambda + 2G)e_{,z} + 2G(\Omega_{x,y} - \Omega_{y,x}) &= -\rho \omega^2 w
    \end{align}
\end{subequations}

Onde a constante de Lamé $\lambda$ é expressa em função do módulo de elasticidade $E$ e do coeficiente de Poisson $\nu$ como:

\begin{equation}
    \lambda = \frac{\nu E}{(1+\nu)(1-2\nu)} \label{eq:lame}
\end{equation}

Para desacoplar essas equações e revelar os tipos de onda, realizam-se operações diferenciais específicas. 

Primeiramente, para isolar a deformação volumétrica, derivam-se as três equações acima em relação a $x, y$ e $z$, respectivamente, e somam-se os resultados.
Isso elimina os termos rotacionais, resultando na equação da onda dilatacional:

\begin{equation}
    (\lambda + 2G)(e_{,xx} + e_{,yy} + e_{,zz}) = -\rho \omega^2 e \label{eq:onda_p_geral}
\end{equation}

Utilizando o operador Laplaciano ($\nabla^2$), esta equação é reescrita de forma compacta:

\begin{equation}
    \nabla^2 e = - \frac{\omega^2}{c_p^2} e \label{eq:helmholtz_e}
\end{equation}

Aqui surge naturalmente a definição da velocidade da onda dilatacional (ou onda P), $c_p$, que depende exclusivamente das propriedades do material:

\begin{equation}
    \label{eq:cp}
    c_p^2 = \frac{\lambda + 2G}{\rho}
\end{equation}

De forma análoga, para isolar as rotações,
elimina-se a deformação volumétrica das equações de movimento (tomando o rotacional do campo ou subtraindo as derivadas cruzadas).
Isso leva às equações governantes para as componentes de rotação:

\begin{subequations}
    \label{eq:onda_s_componentes}
    \begin{align}
        G(\Omega_{x,xx} + \Omega_{x,yy} + \Omega_{x,zz}) &= -\rho \omega^2 \Omega_x \\
        G(\Omega_{y,xx} + \Omega_{y,yy} + \Omega_{y,zz}) &= -\rho \omega^2 \Omega_y \\
        G(\Omega_{z,xx} + \Omega_{z,yy} + \Omega_{z,zz}) &= -\rho \omega^2 \Omega_z
    \end{align}
\end{subequations}

Introduzindo $c_s$, identificada como a velocidade da onda de cisalhamento (ou onda S):

\begin{equation}
    \label{eq:cs}
    c_s^2 = \frac{G}{\rho}
\end{equation}

Pode-se escrever a equação vetorial para o campo de rotações $\{\Omega\}$:

\begin{equation}
    \nabla^2 \{\Omega\} = - \frac{\omega^2}{c_s^2} \{\Omega\} \label{eq:helmholtz_omega}
\end{equation}

As equações obtidas são equações diferenciais parciais lineares de segunda ordem. A primeira governa a propagação de ondas associadas a mudanças de volume (Ondas P),
propagando-se com velocidade $c_p$. A segunda governa a propagação de ondas associadas à distorção sem mudança de volume (Ondas S), propagando-se com velocidade $c_s$.

\subsection{Onda P (P-Wave)}

Agora que isolamos a equação para a deformação volumétrica $e$ \eqref{eq:helmholtz_e}, podemos resolvê-la. Esta é uma equação de onda escalar clássica.

Para encontrar a solução, assumimos uma \textbf{onda plana} se propagando no espaço. Matematicamente, representamos isso por uma função exponencial complexa.
A escolha dessa função não é arbitrária; ela representa uma perturbação que varia senoidalmente tanto no tempo quanto no espaço.

A solução tentativa para a amplitude da deformação volumétrica é:

\begin{equation}
    e = -\frac{i\omega}{c_p} A_p \exp\left[\frac{i\omega}{c_p}(-l_x x - l_y y - l_z z)\right] \label{eq:solucao_onda_p}
\end{equation}

Vamos "traduzir" os termos desta expressão:
\begin{itemize}
    \item \textbf{$A_p$}: É a amplitude da onda (o "tamanho" máximo da deformação).
    \item \textbf{$l_x, l_y, l_z$}: São os \textit{cossenos diretores}. Eles funcionam como coordenadas de um vetor unitário que aponta para onde a onda está indo.
    Para garantir que seja um vetor unitário (comprimento 1), deve-se satisfazer:
    \begin{equation}
        l_x^2 + l_y^2 + l_z^2 = 1 \label{eq:vetor_unitario_p}
    \end{equation}
    \item \textbf{O termo exponencial}: Representa a fase da onda. A expressão $(-l_x x - l_y y - l_z z)$ define frentes de onda planas perpendiculares à direção de propagação.
\end{itemize}

O termo $s = l_x x + l_y y + l_z z$ pode ser interpretado como a distância percorrida ao longo da linha de propagação.
Assim, para um tempo fixo, a deformação é constante em qualquer plano perpendicular a essa direção $s$.

\subsection*{Cálculo dos Deslocamentos}
Uma vez definida a deformação volumétrica $e$, podemos recuperar os deslocamentos físicos $u, v, w$. O resultado matemático revela a natureza física da Onda P:

\begin{subequations}
\label{eq:deslocamentos_p}
\begin{align}
    u_p &= l_x A_p \exp\left[\frac{i\omega}{c_p}(-l_x x - l_y y - l_z z)\right] \\
    v_p &= l_y A_p \exp\left[\frac{i\omega}{c_p}(-l_x x - l_y y - l_z z)\right] \\
    w_p &= l_z A_p \exp\left[\frac{i\omega}{c_p}(-l_x x - l_y y - l_z z)\right]
\end{align}
\end{subequations}

\textbf{Interpretação Física Importante:}
Observe que o deslocamento em cada direção ($u, v, w$) é proporcional ao respectivo cosseno diretor ($l_x, l_y, l_z$) que define a direção da onda.
Isso significa que o \textbf{vetor deslocamento resultante é paralelo à direção de propagação}.

É por isso que a Onda P é chamada de onda dilatacional ou longitudinal:
o movimento da partícula ocorre "para frente e para trás" ao longo da mesma linha em que a onda viaja, comprimindo e expandindo o material (daí o termo "dilatacional"),
sem causar rotação ou distorção de forma.

\subsection{Onda S (S-Wave)}

Enquanto a Onda P está associada à mudança de volume, a Onda S (Onda Secundária ou de Cisalhamento) está associada à distorção.
A equação que governa este fenômeno \eqref{eq:helmholtz_omega} é vetorial, o que torna a solução um pouco mais rica:
o movimento da partícula pode ocorrer em qualquer direção, desde que seja perpendicular à propagação.

Para resolver a equação $\nabla^2 \{\Omega\} = - \frac{\omega^2}{c_s^2} \{\Omega\}$, propomos uma solução análoga à da Onda P, mas agora para o vetor de rotação:

\begin{equation}
    \{\Omega\} = -\frac{i\omega}{2c_s} \{C\} \exp\left[\frac{i\omega}{c_s}(-m_x x - m_y y - m_z z)\right] \label{eq:solucao_onda_s}
\end{equation}

Onde:
\begin{itemize}
    \item $c_s$ é a velocidade da onda de cisalhamento.
    \item $m_x, m_y, m_z$ são os cossenos diretores da propagação da Onda S (análogos aos $l_i$ da Onda P).
    \item $\{C\}$ é um vetor de amplitude constante.
\end{itemize}

Existe uma restrição física importante: como o campo é rotacional (divergência nula), o vetor de amplitude $\{C\}$ deve ser perpendicular à direção de propagação.
Matematicamente:
\begin{equation}
    m_x C_x + m_y C_y + m_z C_z = 0 \label{eq:condicao_ortogonalidade}
\end{equation}

Calculando os deslocamentos resultantes ($u_s, v_s, w_s$) a partir das rotações, percebe-se que o movimento da partícula ocorre num plano \textbf{perpendicular} à direção de propagação.
É por isso que a Onda S é chamada de \textit{onda transversal}.

\subsection*{Decomposição em Ondas SH e SV}
Como o movimento da partícula é perpendicular à propagação, ele pode "apontar" para várias direções.
Para facilitar a análise em engenharia civil (especialmente em solos estratificados horizontalmente), decompomos a Onda S em duas componentes ortogonais:

\begin{enumerate}
    \item \textbf{Onda SH (Horizontal):} A componente do movimento que ocorre estritamente no plano horizontal ($x-y$).
    Esta onda não causa deslocamento vertical ($w_s = 0$). É fundamental em terremotos pois "sacode" as estruturas lateralmente.
    
    \item \textbf{Onda SV (Vertical):} A componente do movimento que reside no plano vertical definido pela direção de propagação e o eixo $z$.
    Note que, apesar do nome, ela não é puramente vertical; ela tem componentes verticais e horizontais, mas "vive" num plano vertical.
\end{enumerate}

Se definirmos as amplitudes dessas componentes como $A_{SH}$ e $A_{SV}$, os deslocamentos físicos podem ser escritos de forma mais intuitiva:

\begin{subequations}
\label{eq:deslocamentos_sh_sv}
\begin{align}
    u_s &= \frac{m_x m_z A_{SV} - m_y A_{SH}}{\sqrt{m_x^2 + m_y^2}} E_s \\
    v_s &= \frac{m_y m_z A_{SV} + m_x A_{SH}}{\sqrt{m_x^2 + m_y^2}} E_s \\
    w_s &= -\sqrt{m_x^2 + m_y^2} A_{SV} E_s
\end{align}
\end{subequations}

Onde o termo exponencial de fase foi abreviado para:

$$E_s = \exp\left[\frac{i\omega}{c_s}(-m_x x - m_y y - m_z z)\right]$$

\textbf{Interpretação:}
\begin{itemize}
    \item Observe a equação de $w_s$: ela depende apenas de $A_{SV}$. Isso confirma que a onda SH ($A_{SH}$) não gera movimento vertical.
    \item Para uma onda incidindo verticalmente (vindo diretamente de baixo para cima, onde $m_x=0, m_y=0$), essa distinção geométrica perde o sentido clássico,
    e tratamos apenas como componentes horizontais $u$ e $v$.
\end{itemize}

\subsection{Amortecimento Material}

Até agora, assumimos um meio perfeitamente elástico, onde a energia da onda se conserva indefinidamente.
No entanto, solos reais dissipam energia devido ao atrito intergranular e outros mecanismos internos. Esse fenômeno é chamado de amortecimento material ou histerético.

Para incluir esse efeito nas equações sem ter que reescrever toda a teoria, utilizamos o \textbf{Princípio da Correspondência}.
A ideia é simples: substituímos as constantes elásticas reais por \textit{constantes complexas}.
A parte real continua representando a rigidez (armazenamento de energia), enquanto a parte imaginária representa a dissipação.

Definimos os módulos complexos (denotados por um asterisco) da seguinte forma:

\begin{subequations}
\label{eq:modulos_complexos}
\begin{align}
    \lambda^* + 2G^* &= (\lambda + 2G)(1 + 2i\zeta_p) \\
    G^* &= G(1 + 2i\zeta_s)
\end{align}
\end{subequations}

Onde:
\begin{itemize}
    \item $\zeta_p$ é a razão de amortecimento histerético para Ondas P.
    \item $\zeta_s$ é a razão de amortecimento histerético para Ondas S.
\end{itemize}

Como as velocidades de onda dependem desses módulos (\eqref{eq:cp} e \eqref{eq:cs}), elas também se tornam complexas:

\begin{subequations}
\label{eq:velocidades_complexas}
\begin{align}
    c_p^* &= c_p \sqrt{1 + 2i\zeta_p} \\
    c_s^* &= c_s \sqrt{1 + 2i\zeta_s}
\end{align}
\end{subequations}

Essa abordagem matemática é poderosa: todas as equações que deduzimos anteriormente para o caso elástico continuam válidas para o caso amortecido,
bastando substituir $c_p$ e $c_s$ por $c_p^*$ e $c_s^*$.

\subsection{Movimento Total (Combinação de Ondas)}

Na prática da engenharia geotécnica, é comum assumir que as ondas se propagam num plano vertical específico (por exemplo, o plano $x-z$).
Isso simplifica a análise, pois podemos fixar a direção horizontal ($y$) sem perder generalidade.

Se assumirmos que a propagação ocorre no plano $x-z$, os cossenos diretores na direção $y$ tornam-se nulos ($l_y = m_y = 0$).
Somando as contribuições das Ondas P, SV e SH, e incorporando o amortecimento (usando as velocidades complexas),
chegamos às expressões finais para o movimento total em um ponto:

\begin{subequations}
\label{eq:movimento_total}
\begin{align}
    u &= \underbrace{l_x A_p E_p}_{\text{Onda P}} + \underbrace{m_z A_{SV} E_s}_{\text{Onda SV}} \\
    v &= \underbrace{A_{SH} E_s}_{\text{Onda SH}} \\
    w &= \underbrace{l_z A_p E_p}_{\text{Onda P}} - \underbrace{m_x A_{SV} E_s}_{\text{Onda SV}}
\end{align}
\end{subequations}

Onde os termos de fase complexa são:
\begin{align*}
    E_p &= \exp\left[i\omega\left(-\frac{l_x x}{c_p^*} - \frac{l_z z}{c_p^*}\right)\right] \\
    E_s &= \exp\left[i\omega\left(-\frac{m_x x}{c_s^*} - \frac{m_z z}{c_s^*}\right)\right]
\end{align*}

\textbf{Conclusão Fundamental da Seção 5.2:}
As equações \eqref{eq:movimento_total} revelam um desacoplamento físico crucial para problemas planos ($x-z$):
\begin{itemize}
    \item O deslocamento fora do plano ($v$) depende \textbf{exclusivamente} da Onda SH. Isso permite analisar o movimento antiplano (out-of-plane) separadamente.
    \item Os deslocamentos no plano ($u$ e $w$) dependem de uma combinação acoplada de Ondas P e SV. Isso forma o problema do movimento no plano (in-plane).
\end{itemize}

Essa separação entre o problema \textit{antiplano} (SH) e o problema \textit{no plano} (P-SV) será a base para a construção das matrizes de rigidez dinâmica nas próximas seções.

\section{Matriz de Rigidez Dinâmica para Movimento Fora do Plano}

Nesta seção, aplicamos a teoria de ondas desenvolvida anteriormente para modelar uma camada de solo horizontal.
O objetivo é tratar essa camada como um elemento estrutural,
derivando uma matriz de rigidez que relacione as forças aplicadas nas interfaces (topo e base) com os deslocamentos resultantes.

Começamos pelo movimento fora do plano (associado às Ondas SH), pois é matematicamente mais simples:
envolve apenas o deslocamento transversal $v$ e não acopla com as outras direções.

\subsection{Tipos de Ondas}

Considere uma camada de solo homogênea com profundidade $d$. Diferente de um meio infinito, aqui as ondas batem nas interfaces e refletem.
Portanto, a solução geral para o deslocamento $v(z,x)$ deve ser a soma de duas ondas:
\begin{itemize}
    \item Uma onda incidente (descendo, amplitude $A_{SH}$);
    \item Uma onda refletida (subindo, amplitude $B_{SH}$).
\end{itemize}

A equação do movimento, adaptada para incluir a reflexão, torna-se:

\begin{equation}
    v(z,x) = \underbrace{\left[ A_{SH} \exp(iktz) + B_{SH} \exp(-iktz) \right]}_{\text{Variação com a profundidade } z} \exp(-ikx) \label{eq:campo_deslocamento_sh}
\end{equation}

Para simplificar a notação, introduzimos as variáveis auxiliares fundamentais para a formulação matricial:
\begin{itemize}
    \item $k = \omega/c$: número de onda (frequência espacial horizontal);
    \item $c = c_s^*/m_x$: velocidade de fase aparente na direção horizontal;
    \item $t = \sqrt{(c/c_s^*)^2 - 1}$: um escalar que relaciona a propagação vertical com a horizontal (ligado ao ângulo de incidência).
\end{itemize}

\subsection{Matrizes de Transferência e de Rigidez Dinâmica de Camada e de Semiespaço}

Para a engenharia, não estamos interessados nas amplitudes de onda $A_{SH}$ e $B_{SH}$ (que são conceitos abstratos),
mas sim nos deslocamentos $v$ e tensões $\tau_{yz}$ nas fronteiras da camada.

\subsubsection*{Matriz da Camada (Layer)}

\begin{enumerate}
    \item \textbf{Cálculo das Tensões:} Usando a Lei de Hooke ($\tau_{yz} = G^* v_{,z}$), derivamos a Eq. \eqref{eq:campo_deslocamento_sh} em relação a $z$:
    \begin{equation}
        \tau_{yz}(z) = iktG^* \left[ A_{SH} \exp(iktz) - B_{SH} \exp(-iktz) \right] \label{eq:tensao_sh}
    \end{equation}
    
    \item \textbf{Eliminação das Amplitudes:} Aplicamos as equações no topo ($z=0$) e na base ($z=d$). Isolando $A_{SH}$ e $B_{SH}$,
    reorganizamos o sistema para que os deslocamentos ($v_1, v_2$) sejam as variáveis independentes e as forças nodais ($Q_1, Q_2$) sejam as dependentes.
    
    \textit{Nota: As forças externas $Q$ equilibram as tensões internas, logo definimos $Q_1 = -\tau_{yz1}$ (topo) e $Q_2 = +\tau_{yz2}$ (base).}
\end{enumerate}

O resultado final é a \textbf{Matriz de Rigidez Dinâmica da Camada} $[S_{SH}^L]$:

\begin{equation}
    \begin{Bmatrix} Q_1 \\ Q_2 \end{Bmatrix} = 
    \frac{ktG^*}{\sin(ktd)} 
    \begin{bmatrix} 
        \cos(ktd) & -1 \\ 
        -1 & \cos(ktd) 
    \end{bmatrix} 
    \begin{Bmatrix} v_1 \\ v_2 \end{Bmatrix} \label{eq:matriz_rigidez_sh_layer}
\end{equation}

Esta matriz $2 \times 2$ funciona exatamente como a matriz de rigidez de uma barra em análise estrutural clássica,
mas seus termos dependem da frequência $\omega$ (através de $k$ e $t$).

\subsubsection*{Matriz do Semiespaço (Half-Space)}

Frequentemente, a última camada do solo se estende infinitamente para baixo (rocha basal ou bedrock). Este é o "Semiespaço".
A diferença física crucial é a \textbf{Condição de Radiação}: ondas podem descer para o infinito, mas nenhuma onda retorna de lá.

Isso implica que não existe onda refletida, logo $B_{SH} = 0$.
Substituindo isso nas equações de topo ($z=0$) e eliminando $A_{SH}$, obtemos uma relação direta e simples entre a força e o deslocamento na superfície:

\begin{equation}
    Q_0 = \underbrace{(iktG^*)}_{S_{SH}^R} v_0 \label{eq:rigidez_sh_halfspace}
\end{equation}

O coeficiente de rigidez do semiespaço é:
\begin{equation}
    S_{SH}^R = iktG^* \label{eq:coef_rigidez_sh_r}
\end{equation}

\textbf{Interpretação Física:}
Para um meio elástico (sem amortecimento material), $S_{SH}^R$ é puramente imaginário (devido ao termo $i$).
Em dinâmica estrutural, uma rigidez imaginária ($i\omega C$) representa um \textbf{amortecedor (dashpot)}.
Isso significa que o semiespaço age como um "sumidouro de energia", dissipando a energia vibratória para o infinito através da radiação geométrica.

\subsection{Casos Especiais}

As expressões gerais derivadas anteriormente (Eqs. \ref{eq:matriz_rigidez_sh_layer} e \ref{eq:coef_rigidez_sh_r}) podem apresentar indefinições matemáticas ou simplificações importantes em certas situações limite.
Estes casos são essenciais para verificação numérica e compreensão física.

\begin{description}
    \item[1. Incidência Vertical ($\omega > 0, k=0$):]
    Este é o caso mais importante para a engenharia sísmica prática.
    Quando $k=0$, o comprimento de onda horizontal é infinito,
    o que implica que a onda atinge toda a superfície horizontal simultaneamente (propagação puramente vertical).
    
    Neste limite, $kt$ converge para $\omega/c_s^*$. A matriz de rigidez da camada torna-se:
    \begin{equation}
        [S_{SH}^L] = \frac{G^*}{c_s^* \sin(\frac{\omega d}{c_s^*})} \cdot \omega
        \begin{bmatrix} 
            \cos(\frac{\omega d}{c_s^*}) & -1 \\ 
            -1 & \cos(\frac{\omega d}{c_s^*}) 
        \end{bmatrix}
    \end{equation}
    
    Para o semiespaço, a rigidez torna-se:
    \begin{equation}
        S_{SH}^R = i G^* \frac{\omega}{c_s^*} = i \omega (\rho c_s^*)
    \end{equation}
    \textbf{Interpretação:} O termo $\rho c_s^*$ é a impedância específica do meio.
    Como o resultado é puramente imaginário e proporcional à frequência ($i\omega C$),
    o semiespaço comporta-se como um \textbf{amortecedor viscoso puro}. Este é o fundamento dos modelos de fronteira absorvente (como o \textit{Dashpot de Lysmer}).

    \item[2. Caso Estático ($\omega = 0, k \neq 0$):]
    Representa a resposta do solo a cargas estáticas que variam espacialmente (harmonicamente ao longo de $x$).
    Neste caso, os termos trigonométricos tornam-se hiperbólicos:
    \begin{equation}
        [S_{SH}^L] = \frac{k G^*}{\sinh(kd)} 
        \begin{bmatrix} 
            \cosh(kd) & -1 \\ 
            -1 & \cosh(kd) 
        \end{bmatrix}
    \end{equation}
    Para o semiespaço, $S_{SH}^R = k G^*$. Note que a rigidez agora é real (mola),
    pois sem frequência ($\omega=0$) não há radiação de energia nem amortecimento histerético (se $G^*$ for real).

    \item[3. Alta Frequência ou Ondas Curtas ($k \to \infty$):]
    Quando o número de onda é muito alto (comprimento de onda muito curto em relação à espessura da camada), a interação entre o topo e a base desaparece.
    O termo fora da diagonal na matriz de rigidez tende a zero,
    significando que o topo da camada "não enxerga" a base e se comporta como se estivesse sobre um semiespaço.
\end{description}

\subsection{Camada Carregada (Loaded Layer)}

Até o momento, consideramos que as forças externas atuam apenas nas interfaces (nós) da camada.
No entanto, em problemas reais (como o peso próprio do solo ou interação estaca-solo),
podem existir cargas distribuídas $q(z)$ atuando ao longo da profundidade da camada.

A estratégia de solução é análoga ao Método dos Deslocamentos na análise de estruturas convencional:
\begin{enumerate}
    \item \textbf{Estado Bloqueado (Solução Particular):} Consideramos as interfaces fixa (deslocamentos nulos) e calculamos as reações de engastamento perfeito geradas pela carga distribuída.
    \item \textbf{Estado Livre (Solução Homogênea):} Aplicamos o inverso dessas reações no sistema total para liberar as interfaces.
\end{enumerate}

Considerando uma variação linear da carga ao longo da profundidade (indo de $q_1$ no topo a $q_2$ na base),
a equação de equilíbrio dinâmico ganha um termo de força extra:

\begin{equation}
    \tau_{yx,x} + \tau_{yz,z} = -\rho\omega^2 v - q(z) \label{eq:equilibrio_com_carga}
\end{equation}

A solução final é a soma da solução particular $v^p$ (devida à carga) com a solução homogênea $v^h$ (devida aos deslocamentos de contorno).
As forças nodais equivalentes ($Q_1, Q_2$) a serem aplicadas na matriz de rigidez global são as reações de engastamento com sinal trocado:

\begin{equation}
    Q_{total} = -Q^p - Q^h \label{eq:superposicao_cargas}
\end{equation}

Onde $Q^p$ são as reações calculadas fixando-se as bordas (Eq. 5.117 do livro).

\subsection{Taxa de Transmissão de Energia}

Para entender como a energia se dissipa ou se propaga através do solo, calculamos a taxa de transmissão de energia horizontal $N$ através de uma seção vertical da camada.

Fisicamente, a potência (taxa de energia) é o produto da Força pela Velocidade.
No nosso caso, integramos o trabalho realizado pela tensão de cisalhamento horizontal $\tau_{yx}$ ao longo da profundidade $d$,
tirando a média em um período de vibração $T = 2\pi/\omega$:

\begin{equation}
    N = -\frac{\omega}{2\pi} \int_{0}^{d} \int_{0}^{2\pi/\omega} \text{Re}[\tau_{yx}(z) e^{i\omega t}] \text{Re}[\dot{v}(z) e^{i\omega t}] \, dt \, dz \label{eq:definicao_energia}
\end{equation}

Resolvendo a integral temporal e substituindo as expressões das ondas, chegamos a uma fórmula compacta para o fluxo de energia:

\begin{equation}
    N = \frac{\omega}{2} \text{Re}(kG^*) \int_{0}^{d} |v(z)|^2 dz \label{eq:energia_final_sh}
\end{equation}

\textbf{Interpretação:}
Esta equação mostra que a energia transmitida horizontalmente depende diretamente da parte real de $kG^*$.
Isso confirma que o amortecimento e a rigidez do material controlam quão longe a vibração consegue viajar lateralmente a partir da fonte.

\section{Matriz de Rigidez Dinâmica para Movimento no Plano}

Diferente do movimento fora do plano (onde apenas a onda SH existia de forma isolada),
o movimento no plano envolve a interação simultânea de Ondas P (Dilatacionais) e Ondas SV (Cisalhamento Vertical).

Isso ocorre devido ao fenômeno de \textit{conversão de modo}: quando uma Onda P incide numa interface horizontal,
as condições de contorno não podem ser satisfeitas apenas com uma reflexão P; é necessário gerar também uma onda SV refletida (e vice-versa).
Portanto, as duas ondas estão fisicamente acopladas e devem ser formuladas em conjunto.

\subsection{Tipos de Ondas}

Para modelar a camada de solo, precisamos considerar quatro componentes de onda viajando simultaneamente:
\begin{enumerate}
    \item Onda P descendo (incidente, amplitude $A_P$);
    \item Onda P subindo (refletida, amplitude $B_P$);
    \item Onda SV descendo (incidente, amplitude $A_{SV}$);
    \item Onda SV subindo (refletida, amplitude $B_{SV}$).
\end{enumerate}

Para que essas ondas coexistam e satisfaçam as condições de contorno em toda a extensão da interface (para qualquer $x$),
elas devem viajar horizontalmente com a mesma velocidade de fase aparente $c$. Isso impõe a seguinte relação geométrica (Lei de Snell):

\begin{equation}
    c = \frac{c_p^*}{l_x} = \frac{c_s^*}{m_x} \label{eq:snell_law}
\end{equation}

Os deslocamentos no plano ($u$ e $w$) são a superposição das contribuições de todas essas ondas.
Matematicamente, as equações são expandidas para incluir as ondas refletidas, resultando em expressões que dependem de quatro amplitudes desconhecidas.

Para simplificar a notação, definimos os parâmetros auxiliares $s$ e $t$ (associados às tangentes dos ângulos de incidência):

\begin{equation}
    s = -i\sqrt{1 - \frac{1}{l_x^2}} \quad \text{e} \quad t = -i\sqrt{1 - \frac{1}{m_x^2}}
\end{equation}

\subsection{Matrizes de Transferência e de Rigidez Dinâmica de Camada e de Semiespaço}

A formulação matricial segue a lógica anterior, mas agora a dimensão do problema dobra.
O estado em qualquer profundidade $z$ é definido por quatro variáveis: dois deslocamentos ($u, w$) e duas tensões ($\tau_{xz}, \sigma_z$).

\subsubsection*{Matriz da Camada (Layer)}

Relacionando as forças nodais ($P, R$) com os deslocamentos nodais ($u, w$) no topo e na base, obtemos a matriz de rigidez da camada $[S_{P-SV}^L]$.

\textit{Truque de Simetria:} O livro utiliza um artifício matemático importante:
multiplica-se os deslocamentos verticais $w$ e as forças verticais $R$ pela unidade imaginária $i$.
Isso torna a matriz de rigidez simétrica, facilitando a implementação numérica.

A relação força-deslocamento para a camada assume a forma $4 \times 4$:

\begin{equation}
    \begin{Bmatrix} P_1 \\ iR_1 \\ P_2 \\ iR_2 \end{Bmatrix} = 
    [S_{P-SV}^L]
    \begin{Bmatrix} u_1 \\ iw_1 \\ u_2 \\ iw_2 \end{Bmatrix} \label{eq:sistema_camada_psv}
\end{equation}

A matriz completa $[S_{P-SV}^L]$ é apresentada na Tabela 5-3 do livro e envolve termos trigonométricos complexos dependentes de $ksd$ e $ktd$.

\subsubsection*{Matriz do Semiespaço (Half-Space)}

Para o semiespaço (rocha basal), aplicamos novamente a \textbf{Condição de Radiação}: apenas ondas que descem ($A_P, A_{SV}$) são permitidas; as ondas que sobem ($B_P, B_{SV}$) são nulas.

Isso simplifica o sistema para uma matriz $2 \times 2$ que relaciona as forças na superfície ($P_0, R_0$) com os deslocamentos na superfície ($u_0, w_0$).
O resultado é a matriz de rigidez do semiespaço $[S_{P-SV}^R]$:

\begin{equation}
    \begin{Bmatrix} P_0 \\ iR_0 \end{Bmatrix} = 
    kG^* \begin{bmatrix} 
        S_{xx} & S_{xz} \\ 
        S_{zx} & S_{zz} 
    \end{bmatrix} 
    \begin{Bmatrix} u_0 \\ iw_0 \end{Bmatrix} \label{eq:rigidez_psv_halfspace}
\end{equation}

Os coeficientes desta matriz representam a rigidez e o amortecimento de radiação acoplados.
\begin{itemize}
    \item $S_{xx}$: Rigidez horizontal (resistência ao balanço lateral).
    \item $S_{zz}$: Rigidez vertical (resistência ao assentamento dinâmico).
    \item $S_{xz}, S_{zx}$: Termos de acoplamento (uma força horizontal gera um movimento vertical e vice-versa).
\end{itemize}

\subsection{Casos Especiais}

Alguns casos limite fornecem insights físicos importantes e simplificam as equações.

\begin{description}
    \item[1. Incidência Vertical ($\omega > 0, k=0$):]
    Neste caso, o comprimento de onda horizontal é infinito ($c = \infty$).O acoplamento entre P e SV desaparece.
    \begin{itemize}
        \item A Onda P gera apenas movimento vertical ($w$), governado por $c_p^*$.
        \item A Onda SV gera apenas movimento horizontal ($u$), governado por $c_s^*$.
    \end{itemize}
    A matriz de rigidez do semiespaço torna-se diagonal:
    \begin{equation}
        [S_{P-SV}^R] = i\omega \rho \begin{bmatrix} c_s^* & 0 \\ 0 & c_p^* \end{bmatrix}
    \end{equation}
    Isso confirma que, para incidência vertical, a impedância horizontal é controlada pela velocidade de cisalhamento ($c_s^*$) e a vertical pela velocidade de compressão ($c_p^*$).

    \item[2. Caso Estático ($\omega = 0, k \neq 0$):]
    As equações convergem para a solução da elastostática para uma carga harmônica espacial. Os termos trigonométricos tornam-se hiperbólicos.

    \item[3. Alta Frequência ($k \to \infty$):]
    Como no caso SH, a interação entre topo e base desaparece, e a camada comporta-se como um semiespaço.
\end{description}

\subsection{Camada Carregada (Loaded Layer)}

Assim como no caso SH, podemos ter cargas distribuídas atuando no interior da camada (devido a peso próprio,
forças de inércia de massas inclusas ou interação estaca-solo). No movimento no plano, essas cargas têm duas componentes:
\begin{itemize}
    \item $p(z)$: Carga distribuída horizontal (direção $x$);
    \item $r(z)$: Carga distribuída vertical (direção $z$).
\end{itemize}

Assumindo uma variação linear dessas cargas com a profundidade (de $p_1, r_1$ no topo a $p_2, r_2$ na base),
as equações de equilíbrio dinâmico ganham termos de força não-homogêneos:

\begin{subequations}
\label{eq:equilibrio_carga_plano}
\begin{align}
    \sigma_{x,x} + \tau_{xz,z} &= -\rho\omega^2 u - p(z) \\
    \tau_{zx,x} + \sigma_{z,z} &= -\rho\omega^2 w - r(z)
\end{align}
\end{subequations}

A estratégia de solução é novamente análoga ao \textbf{Método dos Deslocamentos}:
\begin{enumerate}
    \item \textbf{Solução Particular (Estado Bloqueado):} Calculamos a resposta da camada assumindo que as interfaces superior e inferior estão perfeitamente engastadas ($u=w=0$ nas bordas).
    Isso gera reações de apoio nas interfaces ($P^p, R^p$).
    \item \textbf{Superposição:} Para obter a resposta real, aplicamos essas reações com sinal invertido nas interfaces do sistema global (Eq. \ref{eq:sistema_camada_psv}).
\end{enumerate}

As amplitudes das reações de engastamento perfeito ($P^p, R^p$) são obtidas resolvendo o sistema de equações diferenciais para a variação linear de carga,
resultando em expressões algébricas (Eqs. 5.146 do livro) que dependem da rigidez complexa e da frequência.

\subsection{Taxa de Transmissão de Energia}

A análise do fluxo de energia é mais rica no caso do movimento no plano do que no caso SH.
Agora, a energia pode ser transmitida horizontalmente através de dois mecanismos distintos:
\begin{enumerate}
    \item \textbf{Trabalho Normal:} A tensão normal horizontal $\sigma_x$ realizando trabalho sobre a velocidade horizontal $\dot{u}$.
    \item \textbf{Trabalho de Cisalhamento:} A tensão de cisalhamento vertical $\tau_{zx}$ realizando trabalho sobre a velocidade vertical $\dot{w}$.
\end{enumerate}

A taxa de transmissão de energia horizontal $N$ é calculada integrando a soma dessas potências ao longo da profundidade da camada $d$ e tirando a média temporal num período $T = 2\pi/\omega$:

\begin{equation}
    N = -\frac{1}{2} \int_{0}^{d} \text{Re} \left[ \sigma_x(z) \overline{\dot{u}(z)} + \tau_{zx}(z) \overline{\dot{w}(z)} \right] dz \label{eq:energia_plano}
\end{equation}

(Nota: A barra sobre a velocidade indica o conjugado complexo, uma forma padrão de calcular potência média em fasores, equivalente à formulação integral do livro).

\textbf{Interpretação Física:}
Esta equação mostra que a atenuação da vibração com a distância depende de como as ondas P e SV interagem.
Diferente do caso SH, onde só havia cisalhamento, aqui a rigidez à compressão (associada a $\sigma_x$) também joga um papel fundamental no transporte de energia para longe da fonte.